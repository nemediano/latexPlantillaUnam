%%%%%%%%%%%%%%%%%%%%%%%%%%%%%%%%%%%%%%%%%
% Medium Length Professional CV
% LaTeX Template
% Version 2.0 (8/5/13)
%
% This template has been downloaded from:
% http://www.LaTeXTemplates.com
%
% Original author:
% Trey Hunner (http://www.treyhunner.com/)
%
% Important note:
% This template requires the resume.cls file to be in the same directory as the
% .tex file. The resume.cls file provides the resume style used for structuring the
% document.
%
%%%%%%%%%%%%%%%%%%%%%%%%%%%%%%%%%%%%%%%%%

%----------------------------------------------------------------------------------------
%	PACKAGES AND OTHER DOCUMENT CONFIGURATIONS
%----------------------------------------------------------------------------------------

\documentclass{resume} % Use the custom resume.cls style

% custom links and pdf metadata
\usepackage[
  final,
  unicode,
  colorlinks=true,
  citecolor=black,
  linkcolor=black,
  urlcolor=black,
  plainpages=false,
  urlcolor=black,
  pdfpagelabels=true,
  pdfsubject={Curriculum de Pancho Villa},
  pdfauthor={José Doroteo Arango Arámbula},
  pdftitle={Curriculum vitae - Pancho Villa},    % title
  pdfkeywords={Curriculum vitae, Ciencias de la Computación, Doctorado, Graficación por Computadora}, % list of keywords
]{hyperref}

\usepackage[letterpaper,left=1cm,top=1.5cm,right=1cm,bottom=1cm]{geometry} % Document margins

\newcommand{\tab}[1]{\hspace{.2667\textwidth}\rlap{#1}}
\newcommand{\itab}[1]{\hspace{0em}\rlap{#1}}
\name{José Doroteo Arango Arámbula} % Your name
\address{Avenida Alcanfores y San Juan, Totoltepec s/n \\ Sta Cruz Acatlán, 53150} % Your address
\address{\href{http://www.linkedin.com/in/panchovilla}{http://www.linkedin.com/in/panchovilla}} % Dirección secundaria
\address{+52 55 5623 1715 \\ \href{mailto:panchovilla@gmail.com}{panchovilla@gmail.com}} % Teléfono y correo electronico

\usepackage{fancyhdr}
\showboxdepth=5
\showboxbreadth=5

\pagestyle{fancy}
\lhead{}
\chead{}
\rhead{Villa, Pancho pag.\thepage}
\cfoot{} % get rid of the page number 
\renewcommand{\headrulewidth}{0pt}
\renewcommand{\footrulewidth}{0pt}

\begin{document}
\thispagestyle{empty}
%----------------------------------------------------------------------------------------
%	EDUCATION SECTION
%----------------------------------------------------------------------------------------

\begin{rSection}{Educación}

{\bf \href{http://www.purdue.edu}{UNIVERSIDAD DE PURDUE}} \hfill {\em West Lafayette, Indiana, EUA} 
\\ \href{http://polytechnic.purdue.edu/degrees/phd-technology}{{\bf Doctorado} en Tecnología} \hfill {\em Agosto 2017}
\\ \href{http://polytechnic.purdue.edu}{Instituto Politécnico de Purdue}, \href{http://polytechnic.purdue.edu/departments/computer-graphics-technology}{Departamento de Graficación por Computadora} \hfill {\em Promedio: 3.79/4.0}
\\ Áreas de Investigación: Manufactura aditiva (Impresión 3D) y Graficación por Computadora

{\bf \href{http://www.unam.mx}{UNIVERSIDAD NACIONAL AUTÓNOMA DE MÉXICO}} \hfill {\em CDMX} 
\\ \href{http://www.mcc.unam.mx}{{\bf Maestría} en Ciencias de la Computación} \hfill {\em Septiembre 2011}
\\ \href{https://www.iimas.unam.mx}{Instituto de Investigación en Matemáticas Aplicadas y Sistemas} \hfill {\em Promedio: 9.4/10.0}
\\ Areas de estudio: Procesamiento de Imágenes Digitales y Graficación por Computadora

\href{http://www.mac.acatlan.unam.mx}{{\bf Licenciatura} en Matemáticas Aplicadas y Computación} \hfill {\em Mayo 2008}
\\ \href{https://www.acatlan.unam.mx}{Facultad de Estudio Superiores de Acatlán} \hfill {\em Promedio: 8.6/10.0}


\end{rSection}
%----------------------------------------------------------------------------------------
%	TECHNICAL STRENGTHS SECTION
%----------------------------------------------------------------------------------------

\begin{rSection}{Conocimientos en Computación}

\begin{tabular}{ @{} >{\bfseries}l @{\hspace{2ex}} l }
Lenguajes de programación &  \textbf{Avan:} C/C++, GLSL. \textbf{Med:} Matlab, JavaScript. \textbf{Prin:} Python, R\\
APIs / Frameworks & OpenGL, Qt, ASP.NET MVC, CUDA\\
Herramientas de desarrollo &  SVN, Git, \LaTeX, bash, phpESP, Wordpress\\
Paquetería & Visual Studio, Eclipse, Gimp, Inkscape
\end{tabular}

\end{rSection}

%----------------------------------------------------------------------------------------
%	WORK EXPERIENCE SECTION
%----------------------------------------------------------------------------------------

\begin{rSection}{Experiencia en la industria}

\begin{rSubsection}{\href{http://about.google/}{Google LLC}}{Mountain View, California, EUA}{Software Engineer, Equipo de Google Cloud}{Marzo 2020 - presente}
\item Miembro del equipo de tecnología avanzada en \href{http://stadia.google.com/}{Stadia}.
\end{rSubsection}

\begin{rSubsection}{\href{http://www.nvidia.com}{Nvidia Corporation}}{Santa Clara, California, EUA}{Senior Software Engineer, Equipo 3D para dispositivos móviles}{Agosto 2017 - Feb 2021}
\item Contribuí en la certificación de seguridad de una biblioteca (MISRA y CERT C) siguiendo la metodología PLC
\item Realicé dos apps de realidad aumentada para comparar ARCore y ARKit en teléfonos usando Unreal Engine
\end{rSubsection}

\begin{rSubsection}{\href{http://www.adobe.com/}{Adobe Systems Incorporated}}{San Francisco, California, EUA}{Becario de investigación, \href{http://research.adobe.com/}{Equipo de Imágenes Procedurales}}{Mayo 2016 - Agosto 2016}
\item Diseñé un motor gráfico de \emph{rendering} diferido para una aplicación para escultura interactiva
\item Implementé varios algoritmos: \emph{shadow mapping}, \emph{global ambient occlusion} y \emph{PBR shading}, usando GLSL shaders
\end{rSubsection}

\begin{rSubsection}{\href{http://www.nvidia.com}{Nvidia Corporation}}{Santa Clara, California, EUA}{Becario en desarrollo de Software, Equipo del driver de OpenGL}{Mayo 2015 - Agosto 2015}
\item Porté extensiones de OpenGL para exponerlas en el API público.
\item Entable comunicación con ingenieros de diversas partes del mundo para detallar reportes de bugs
\end{rSubsection}

\begin{rSubsection}{Nvidia Corporation}{Santa Clara, California, EUA}{Becario en desarrollo de Software, Equipo del driver de DirectX}{Mayo 2014 - Agosto 2014}
\item Cree un conversor de formato de imágenes digitales, usando el lenguaje ensamblador propietario de Nvidia
\item Escribí varios shaders en HLSL para un conversor de formatos de vídeo
\end{rSubsection}

\end{rSection}

\clearpage



\begin{rSection}{Experiencia en ámbito académico}

	\begin{rSubsection}{\href{http://www.purdue.edu}{Universidad de Purdue}}{West Lafayette, Indiana, EUA}{Ayudante de profesor, Depto de Gráficas por Computadora}{Agosto 2015 - Mayo 2016, Agosto 2016 - Mayo 2017}
	\item Cree materiales didácticos para la clase de programación: \href{https://polytechnic.purdue.edu/sites/default/files/CGT-fall-2017.pdf}{CGT215}
	\item Supervisé estudiantes durante sesiones de laboratorio de clases de programación
	\item Califique material y administré las calificaciones para más de 50 estudiantes usando Blackboard
	\end{rSubsection}

	\begin{rSubsection}{\href{http://www.purdue.edu}{Universidad de Purdue}}{West Lafayette, Indiana, EUA}{Asistente de investigación, \href{http://hpcg.purdue.edu/}{Laboratorio HPCG}}{Agosto 2012 - Mayo 2014, Agosto 2014 - Mayo 2015}
	\item Cree una visualización 3D de las microestructuras internas de las baterías usando OpenGL y CUDA
	\item Desarrolle un algoritmo de optimización de empacado 3D
	\item Contribuí en el desarrollo de software para el análisis de redes carreteras usando teoría de grafos
	\end{rSubsection}
	
	\begin{rSubsection}{\href{http://www.unam.mx}{Universidad Nacional Autónoma de México}}{CDMX, México}{Asistente de investigación, \href{https://turing.iimas.unam.mx/}{Depto. de Ciencias de la Computación}, \href{https://www.iimas.unam.mx/}{IIMAS}}{Octubre 2011 - Enero 2012}
	\item Desarrollé programas de análisis de imágenes digitales para la reconstrucción 3D de venas en la retina
	\item Opere una \emph{cámara fundus} para capturar imágenes de la retina
	\item Cree un pipeline de varios scripts en Bash para automatizar el procesamiento de imágenes
	\end{rSubsection}
	
	\begin{rSubsection}{\href{http://www.unam.mx}{Universidad Nacional Autónoma de México}}{Edo de México, México}{Profesor, \href{http://www.mac.acatlan.unam.mx}{Jefatura de MAC}, \href{https://www.acatlan.unam.mx}{FES Acatlán}} {Agosto 2008 - Mayo 2012}
	\item Impartí dos clases: \href{https://www.acatlan.unam.mx/files/PlanesDeEstudio/MAC/4/Teoria_de_Graficas.pdf}{Teoría de Grafos} y \href{https://www.acatlan.unam.mx/files/PlanesDeEstudio/MAC/7/Graficacion_por_Computadora.pdf}{Graficación por Computadora}
	\item Cree materiales didácticos: diapositivas, exámenes y notas
	\item Dí mentoría a alumnos en horas extracurriculares
	\end{rSubsection}
	
	\begin{rSubsection}{\href{http://www.unam.mx}{Universidad Nacional Autónoma de México}}{CDMX, México}{\href{http://www.fciencias.unam.mx/directorio/63922}{Ayudante de profesor}, \href{http://www.fciencias.unam.mx/}{Facultad de Ciencias}}{Enero 2009 - Diciembre 2010}
	\item Colaboré en dos clases nivel licenciatura: \href{http://www.fciencias.unam.mx/licenciatura/asignaturas/217/249}{Geometría Moderna} e \href{http://www.fciencias.unam.mx/licenciatura/asignaturas/2017/1236}{Introducción a las Ciencias de la Computación}
	\item Cree material didáctico, programas de demostración, exámenes y diapositivas
	\item Guié a estudiantes en la instalación de software y aprender buenas prácticas de programación
	\end{rSubsection}

\end{rSection}

\begin{rSection}{Publicaciones} \itemsep 1pt
\item \textit{\href{http://www.sciencedirect.com/science/article/pii}{Improving troops orientation for avoiding ambush in scarse battles}}, { \bf Arango D.} y Zapata E. Active Guerrila Journal, July 2018.
\item \textit{\href{http://dl.acm.org/doi/abs/10.1145/}{Learning Fast artillery charges}}, Zapata E, { \bf Arango D.}., y Ángeles, F. Proceedings of the 32Nd Spring Conference on Revolutions, 2016.
\end{rSection}

\end{document}
