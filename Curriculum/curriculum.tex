%%%%%%%%%%%%%%%%%%%%%%%%%%%%%%%%%%%%%%%%%
% Medium Length Professional CV
% LaTeX Template
% Version 2.0 (8/5/13)
%
% This template has been downloaded from:
% http://www.LaTeXTemplates.com
%
% Original author:
% Trey Hunner (http://www.treyhunner.com/)
%
% Important note:
% This template requires the resume.cls file to be in the same directory as the
% .tex file. The resume.cls file provides the resume style used for structuring the
% document.
%
%%%%%%%%%%%%%%%%%%%%%%%%%%%%%%%%%%%%%%%%%

%----------------------------------------------------------------------------------------
%	PACKAGES AND OTHER DOCUMENT CONFIGURATIONS
%----------------------------------------------------------------------------------------

\documentclass{resume} % Usar el estilo definido en resume.cls

% Links personalizados (en un documento impreso no se ven)
% y metadatos del pdf, en un curriculum esto es muy importante por que
% es muy probable que el primer filtro sea un sistema

\usepackage[
  final,
  unicode,
  colorlinks=true,
  citecolor=black,
  linkcolor=black,
  urlcolor=black,
  plainpages=false,
  urlcolor=black,
  pdfpagelabels=true,
  pdfsubject={Curriculum de Pancho Villa}, % Tema
  pdfauthor={José Doroteo Arango Arámbula}, % Autor
  pdftitle={Curriculum vitae - Pancho Villa}, % Titulo
  pdfkeywords={Curriculum vitae, Licenciado en Matemáticas Aplicadas y Computación, Graficación por Computadora}, % Keywords
]{hyperref}

\usepackage[letterpaper,left=1cm,top=1.5cm,right=1cm,bottom=1cm]{geometry} % Márgenes del documento

\newcommand{\tab}[1]{\hspace{.2667\textwidth}\rlap{#1}}
\newcommand{\itab}[1]{\hspace{0em}\rlap{#1}}
\name{José Doroteo Arango Arámbula} % El nombre
\address{Avenida Alcanfores y San Juan, Totoltepec s/n \\ Sta Cruz Acatlán, 53150} % Tu dirección
\address{\href{http://www.linkedin.com/in/panchovilla}{http://www.linkedin.com/in/panchovilla}} % Dirección secundaria
\address{+52 55 5623 1715 \\ \href{mailto:panchovilla@gmail.com}{panchovilla@gmail.com}} % Teléfono y correo electronico

\usepackage{fancyhdr}
\showboxdepth=5
\showboxbreadth=5

\pagestyle{fancy}
\lhead{}
\chead{}
%\rhead{Pancho, Villa pag.\thepage}  %En caso de que te pases de una página
\cfoot{} % get rid of the page number 
\renewcommand{\headrulewidth}{0pt}
\renewcommand{\footrulewidth}{0pt}

\begin{document}
\thispagestyle{empty}
%----------------------------------------------------------------------------------------
%	EDUCATION SECTION
%----------------------------------------------------------------------------------------

\begin{rSection}{Educación}

\href{http://www.mac.acatlan.unam.mx}{{\bf Licenciatura} en Matemáticas Aplicadas y Computación} \hfill {\em Mayo 2021}
\\ \href{https://www.acatlan.unam.mx}{Facultad de Estudio Superiores de Acatlán} \hfill {\em Promedio: 8.5/10.0}


\end{rSection}
%----------------------------------------------------------------------------------------
%	TECHNICAL STRENGTHS SECTION
%----------------------------------------------------------------------------------------

\begin{rSection}{Conocimientos en Computación}

\begin{tabular}{ @{} >{\bfseries}l @{\hspace{2ex}} l }
Lenguajes de programación &  \textbf{Avan:} C/C++, GLSL. \textbf{Med:} Matlab, JavaScript. \textbf{Prin:} Python, R\\
APIs / Frameworks & OpenGL, Qt, ASP.NET MVC, CUDA\\
Herramientas de desarrollo &  SVN, Git, \LaTeX, bash, phpESP, Wordpress\\
Paquetería & Visual Studio, Eclipse, Gimp, Inkscape
\end{tabular}

\end{rSection}

%----------------------------------------------------------------------------------------
%	WORK EXPERIENCE SECTION
%----------------------------------------------------------------------------------------

\begin{rSection}{Experiencia en la industria}

\begin{rSubsection}{\href{http://about.google/}{Google LLC}}{Mountain View, California, EUA}{Software Engineer, Equipo de Google Cloud}{Marzo 2020 - presente}
\item Contribuí en la certificación de seguridad de una biblioteca (MISRA y CERT C) siguiendo la metodología PLC
\item Desarrollé varios widgets para web de un framework para robótica (ISAAC SDK)
\end{rSubsection}

\begin{rSubsection}{Tecnología en Sistemas de Software}{CDMX, México}{Becario, Q.A. y desarrollador}{Enero 2019 - Diciembre 2020}
\item Desarrollé varias aplicaciones en Java para procesamiento de texto
\item Ayudé en una migración de una BD de Interbase a Oracle
\end{rSubsection}

\end{rSection}

% Saltar una página
%\clearpage

\begin{rSection}{Experiencia en ámbito académico}

	\begin{rSubsection}{\href{http://www.unam.mx}{Universidad Nacional Autónoma de México}}{Edo de México, México}{Profesor, \href{http://www.mac.acatlan.unam.mx}{Jefatura de MAC}, \href{https://www.acatlan.unam.mx}{FES Acatlán}} {Agosto 2019 - Mayo 2020}
	\item Impartí dos clases: \href{https://www.acatlan.unam.mx/files/PlanesDeEstudio/MAC/4/Teoria_de_Graficas.pdf}{Teoría de Grafos} y \href{https://www.acatlan.unam.mx/files/PlanesDeEstudio/MAC/7/Graficacion_por_Computadora.pdf}{Graficación por Computadora}
	\item Cree materiales didácticos: diapositivas, exámenes y notas
	\end{rSubsection}
	
	\begin{rSubsection}{\href{http://www.unam.mx}{Universidad Nacional Autónoma de México}}{Edo de México, México}{\href{https://www.acatlan.unam.mx/index.php?id=160}{Servicio Social}, \href{https://kalinga.acatlan.unam.mx/cedetec/}{CEDETEC}}{Enero 2019 - Diciembre 2019}
	\item Atendí a usuarios desde ventanilla para asignarles equipos en el área de Linux
	\item Cree material didáctico, programas de demostración, exámenes y diapositivas
	\end{rSubsection}

\end{rSection}

\begin{rSection}{Otras distinciones} \itemsep -6pt
\item Miembro del programa de asesorias para alumnos de primer semestre
\item Miembro del Consejo Técnico de la Facultad, Facultad de Estudios Superiores de Acatlán
\end{rSection}

%----------------------------------------------------------------------------------------
\begin{rSection}{Cursos académicos relevantes e Idiomas}
\itab{Geometría Computacional } \tab{}  \tab{Visualización científica}
\\ \itab{Análisis de Imágenes Multiespectrales } \tab{}  \tab{Series de tiempo} 
\\ \itab{Metaheuristicas para la optimiazación combinatoria } \tab{}  \tab{Procesos estocásticos} 
\\ \itab{Ingles plan Global } \tab{} \tab{TOEFL iBT 110 pts}
\end{rSection}

\end{document}
