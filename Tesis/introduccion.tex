\chapter*{Introducción}
\addcontentsline{toc}{chapter}{Introducción}
\markboth{INTRODUCCIÓN}{INTRODUCCIÓN}

La percepción que tenemos los seres humanos del mundo a través de nuestros sentidos es inmediata. Ademas, dependemos de estos sentidos para realizar la mayoría de las actividades diarias. También hay que señalar que somos criaturas visuales, de nuestros cinco sentidos el que nos proporciona la mayor cantidad de la información que asimilamos sin lugar a dudas son nuestros ojos \cite{Hagen:vista}.

El procesamiento de imágenes estudia como procesar imágenes digitales por medios computacionales \cite{Gonzalez:ImagenesDigitales}. Algunas veces este proceso implica transformar una imagen en otra imagen en donde se ha resaltado, o inhibido, cierta característica. Otras veces implica obtener otro tipo de información de una imagen, por ejemplo un histograma. Finalmente, algunas veces implica poder distinguir o interpretar datos de objetos a partir de las imágenes.

Las gráficas por computadora (GC) son la rama de las CC que estudia las técnicas para producir imágenes digitales a partir de descripciones matemáticas. Estas imágenes son visualizadas en dispositivos de despliegue en 2D, por ejemplo un monitor, por lo que generalmente es necesario hacer una proyección de los objetos (en tres dimensiones) a un plano.

La visualización científica se encarga de construir información visual de conjuntos de datos científicos. Estos datos pueden tener orígenes muy diversos; inclusive pueden simular fenómenos de la naturaleza que no pueden ser percibidos por nuestra vista. Por ejemplo, ver en la pantalla el espacio de solución de una ecuación diferencial que modele la vibración de la cuerda de una guitarra es un problema de visualización científica que se podría equiparar con la idea de ``visualizar la música''.

En éste trabajo se abarca un poco de estas tres áreas. Asumimos que tenemos un conjunto de datos distribuidos en un arreglo cúbico que representan un muestreo de un campo escalar tridimensional. Estos datos forman lo que típicamente se conoce como una imagen digital en 3D o volumen. Debido a la naturaleza de los datos nos interesa poder visualizarlos sin perder su información espacial.

\subsubsection*{Objetivos del trabajo}

El principal objetivo de este trabajo es encontrar por medio de la modificación de normales una forma de hacer la superficie obtenida por el Algoritmo de Artzy visualmente mas agradable. Haciendo enfasis en que no hacemos ninguna suposición sobre como fue la discretización del volumen.
