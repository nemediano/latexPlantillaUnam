\thispagestyle{empty}
\begin{large}
\begin{flushleft}
\includegraphics[scale=0.3]{img/logo-azul}
\end{flushleft}
\begin{flushright}
	\textit{México, D.F., a 20 de noviembre de 2021} \\
	\vspace{0.25cm}
	\textbf{ASUNTO:} Carta de recomendación.
\end{flushright}

\vspace{0.3cm}
\begin{flushleft}
	\textit{Comité Académico del Posgrado en Ciencia e\\Ingeniería de la Computación\\ 			Universidad Nacional Autónoma de México\\}
	\vspace{0.25cm}
	\textbf{P R E S E N T E.}
\end{flushleft}

\vspace{0.3cm}
Es un placer para mí recomendar a \textbf{José Doroteo Arango Arámbula} para el programa de maestría en Ciencias e Ingeniería de la Computación.
El licenciado Arango, fue mi estudiante en la materia de Graficación por Computadora de la FES Acatlán.

Durante ésta clase, –la cual requiere tener conocimientos prácticos y teóricos del campo–, José demostró tener atención a los detalles, compromiso y ser capaz de trabajar con mucha dedicación.
Debido a su gran habilidad en matemáticas y ciencias de la computación, el licenciado Arango también demostró la capacidad de aplicar sus conocimientos teóricos para resolver problemas típicos en el desarrollo de software.
También ha mostrado ser una persona respetuosa, honesta y responsable.

Estoy convencido que José tiene un gran potencial, y que su disciplina y determinación lo llevarán a completar sus estudios de maestría en la Universidad Nacional Autónoma de México.
Si tienen alguna pregunta sobre el desempeño del licenciado Arango, no dude en contactarme a mi correo electrónico: \href{mailto:felipeangeles@unam.mx}{\emph{felipeangeles@unam.mx}}

\bigskip

Sin más por el momento, aprovecho la ocasión para enviarles saludos cordiales.

\vspace{0.2cm}
\begin{flushright}
	\begin{tabular}{ c }
		\textbf{Atentamente}\\
		\includegraphics[scale=0.2]{img/felipeFirma} \\
		Dr. Felipe Ángeles Ramírez \\
		\textit{Profesor de asignatura III} \\
	\end{tabular}
\end{flushright}

\vspace{0.3cm}
%\textbf{Anexo:} Reporte de actividades y plan de trabajo.

\end{large}
